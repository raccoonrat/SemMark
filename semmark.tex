\documentclass[10pt,twocolumn,letterpaper]{article}

\usepackage{usenix2020_SOUPS}
\usepackage{url}
\usepackage{hyperref}
\usepackage{xcolor}
\usepackage{graphicx}
\usepackage{booktabs}
\usepackage{multirow}
\usepackage{amsmath}
\usepackage{amssymb}
\usepackage{enumitem}

% 中文支持 - 使用XeLaTeX编译
\usepackage[UTF8]{ctex}
\usepackage{xeCJK}

% 设置中文字体(如果系统没有这些字体,可以注释掉或替换为系统可用字体)
% \setCJKmainfont{SimSun}
% \setCJKsansfont{SimHei}
% \setCJKmonofont{FangSong}

% 设置超链接颜色
\hypersetup{
    colorlinks=true,
    linkcolor=blue,
    urlcolor=blue,
    citecolor=blue,
    pdfborder={0 0 0}
}

% 页面布局已在 usenix2020_SOUPS.sty 中设置
% 如果需要自定义,可以取消下面的注释
% \setlength{\columnsep}{0.25in}

\title{大模型文本语义水印研究综述:\\
近五年(2021--2025)Top30论文分析与争议点梳理}

\author{匿名作者\\
\and
匿名机构\\
\and
\texttt{anonymous@example.com}
}

\date{}

\begin{document}

\maketitle

\begin{abstract}
本文对近五年(2021--2025)大模型文本语义水印(semantic watermarking for LLM text)及其紧密相关的检测/攻击/理论工作进行系统性综述。我们聚焦Nature/Science/CCS/S\&P/USENIX Security/NDSS/AAAI/NeurIPS/ACL/ICLR等顶级场域,从方法原创性、场域影响力、可复用度(开源工具)、实验透明度四个维度,遴选出Top30论文并展开差异/分歧/矛盾点分析。本文系统梳理了语义层面/后处理水印、工业规模/系统化方案、无偏水印、攻击/跨语种、多比特与公开可验证、安全会议任务面水印等六大主题,并深入分析了关键指标波动超过15\%的争议点,包括检测样本量门槛、多比特可用性、语义vs词面、公开检测API安全性等核心问题。研究揭示了当前领域的主要方法论分歧,并提出了基于统一基准与口径的标准化评估建议。
\end{abstract}

\section{引言}

大模型文本水印技术作为AI内容治理与溯源的重要手段,近年来受到学术界和工业界的广泛关注。本文聚焦\textbf{大模型文本语义水印}(semantic watermarking for LLM text)及其紧密相关的检测/攻击/理论工作,覆盖近五年(2021--2025)在Nature/Science/CCS/S\&P/USENIX Security/NDSS/AAAI/NeurIPS/ACL/ICLR等顶级场域发表的相关研究。

我们以方法原创性、场域影响力、可复用度(开源工具)、实验透明度四维度,遴选\textbf{Top30}核心论文并展开差异/分歧/矛盾点分析与引用排序。部分硬件/系统会议(HPCA/MICRO/ISCA/USENIX ATC/EuroSys)鲜见纯文本水印论文,本文未强行“凑数”,而是专注于文本语义水印的核心研究进展。

\section{Top30核心论文(按主题分组)}

\subsection{语义层面/后处理水印}

语义层面和后处理水印方法更贴近“文本语义水印”的本质,通过句子级语义嵌入或后处理插入实现水印。

\textbf{SemStamp}~\cite{semstamp2024}通过句向量空间的LSH分区+拒绝采样在\textbf{句子级语义}嵌入水印;实证较token级更耐\textbf{释义(paraphrase)}与bigram改写。NAACL 2024(长文)。

\textbf{k-SemStamp}~\cite{ksemstamp2024}以聚类替换LSH,进一步提升采样效率与鲁棒性。ACL 2024(Findings)。

\textbf{SemaMark}~\cite{semamark2024}通过语义替代哈希提升对释义鲁棒性;NAACL 2024(Findings)。

\textbf{PostMark}~\cite{postmark2024}提出\textbf{后处理(post-hoc)语义插入},无需logits访问,第三方可实施;对释义更稳健。EMNLP 2024。

\textbf{Adaptive Text Watermark}~\cite{adaptive2024}通过高熵位点自适应施加水印+语义映射缩放logits,平衡质量与安全性。ICML 2024。

\textbf{Duwak(Dual Watermarks)}~\cite{duwak2024}并行在\textbf{概率分布}与\textbf{采样策略}双通道嵌入密纹,\textbf{检测所需token数可降至既有方法的30\%}(“最多减少70\%”)。ACL 2024(Findings)。

\textbf{GumbelSoft}~\cite{gumbelsoft2024}改进GumbelMax系水印的\textbf{多样性(diversity)}问题,提升AUROC并避免同prompt同输出。ACL 2024。

\textbf{MorphMark}~\cite{morphmark2025}以多目标框架自适应调节水印强度,改善“可检测性$\leftrightarrow$质量”权衡。ACL 2025(Long)。

\subsection{工业规模/系统化方案与基准}

\textbf{SynthID-Text(Google DeepMind)}~\cite{synthid2024}在Nature首发,生产级文本水印与\textbf{推测采样(speculative sampling)}融合;线上近\textbf{2000万}Gemini响应质量评估。Nature 2024;官方开源参考实现。

\textbf{MarkLLM}~\cite{markllm2024}统一实现/可视化/评测管线的\textbf{开源工具包};集成多家方案。EMNLP 2024系统演示。

\textbf{WaterBench}~\cite{waterbench2024}设定“同水印强度”公平对比,联合评估生成/检测,并用GPT-Judge衡量\textbf{质量下降}。ACL 2024。

\textbf{Watermark under Fire(WaterPark)}~\cite{waterpark2025}整合\textbf{12个水印与12类攻击}的鲁棒性评测平台(2025版);揭示设计选择对攻防影响。EMNLP 2025(Findings)。

\subsection{“基线”与分布保持(unbiased)流派}

\textbf{KGW/Green-Red}~\cite{kgw2023}:ICML 2023经典基线;统计检验可公开运行,检测p值可解释。

\textbf{On the Reliability of Watermarks}~\cite{reliability2024}:人机改写后仍可检测;\textbf{FPR=1e-5}下,强人类释义\textbf{需$\sim$800 tokens}观测才稳定检出。ICLR 2024。

\textbf{Unbiased Watermark}~\cite{unbiased2024}:提出“\textbf{分布不扭曲}”水印范式与检测;ICLR 2024。

\textbf{DiPmark}~\cite{dipmark2024}:分布保持+可高效检测的重加权策略。ICML/开放评审稿。

\textbf{MCMARK(Improved Unbiased)}~\cite{mcmark2025}:多通道分割提升无偏水印的可检出性(\textbf{$>$10\%})。ACL 2025(Long)。

\textbf{STA-1(Unbiased \& Low-risk)}~\cite{sta12025}:提出Sampling-Then-Accept一类无偏水印及高效检测。ACL 2025(Long)。

\subsection{攻击/跨语种/可窃取性}

\textbf{Watermarks in the Sand(不可能性)}~\cite{sand2024}:在自然假设下证明“\textbf{强水印}不可实现”,并给出通用去水印随机游走攻击;ICML 2024。

\textbf{Watermark Stealing(ETH)}~\cite{stealing2024}:黑盒逆推水印模式实现\textbf{伪造与去除},实测\textbf{$>$80\%成功率且成本$<$ \$50};ICML 2024。

\textbf{Color-Aware Substitutions(SCTS)}~\cite{scts2024}:\textbf{颜色自测替换}以更少编辑去除KGW水印;可处理任意长文本。ACL 2024。

\textbf{Cross-lingual Consistency(CWRA)}~\cite{cwra2024}:翻译流水线可将AUC\textbf{从0.95降至0.67}(趋近随机);并提出X-SIR防御。ACL 2024。

\textbf{No Free Lunch in LLM Watermarking}~\cite{nofreelunch2024}:系统揭示\textbf{鲁棒性-可用性-可部署性}三难(含多密钥/公开API等);NeurIPS 2024。

\textbf{Attacking by Exploiting Strengths}~\cite{exploiting2024}:把水印“可公开检测”“质量保持”本身视作攻击面;ICLR 2024研讨。

\subsection{多比特与公开可验证/群体追踪}

\textbf{UPV(Unforgeable Publicly Verifiable)}~\cite{upv2024}:生成与检测网络分离、\textbf{可公开验证}而不泄露生成密钥;ICLR 2024。

\textbf{Provably Robust Multi-bit Watermark}~\cite{multibit2025}:段级伪随机分配实现\textbf{多比特追踪};20比特/200 token下\textbf{97.6\%匹配率},SOTA仅\textbf{49.2\%}。USENIX Security 2025。

\textbf{StealthInk(Multi-bit \& Stealth)}~\cite{stealthink2025}:在\textbf{不改分布}前提植入\textbf{多比特溯源信息}(userID/时间戳/模型ID),并给出检测等错误率下token下限。ICML 2025。

\textbf{Multi-User Watermarks}~\cite{multiuser2024}:构造支持\textbf{个体/合谋群体}溯源的多用户水印与统一鲁棒性抽象(AEB-robustness)。IACR ePrint 2024。

\subsection{安全会议的任务面水印/系统化解读}

\textbf{REMARK-LLM(UCSD)}~\cite{remark2024}:面向生成文本的\textbf{学习式编码-重参数化-解码}流水线;\textbf{签名容量$\approx$2$\times$}且对多类攻击更稳。USENIX Security 2024。

\textbf{WaterJudge(质量-检测权衡)}~\cite{waterjudge2024}:提供比较评估框架,挑选“最佳操作点”。NAACL 2024(Findings)。

\textbf{注:}Nature/Science方面,文本水印代表性工作主要是\textbf{SynthID-Text};其余多聚焦多模态/政策评论。USENIX/NDSS/CCS/S\&P侧重\textbf{安全评估/多比特/公开验证/攻击面},而ACL/ICLR/NeurIPS更偏\textbf{算法/理论与鲁棒性评测}的主战场。

\section{数据/理论差异:关键指标波动$>$15\%的争议点}

\subsection{检测样本量(Tokens for Detection)}

\textbf{Duwak}报告在多类后编辑攻击下,为达显著检出,\textbf{所需token数可减少最多70\%},显著优于单通道水印;与传统KGW/Unigram的需求相比形成巨幅落差,直接影响部署门槛与短文本场景可用性。

\subsection{多比特追踪的可靠性(Match/Bit Recovery)}

\textbf{Provably Robust Multi-bit}在\textbf{20比特/200 tokens}场景下\textbf{97.6\%匹配} vs SOTA \textbf{49.2\%},\textbf{差异$>$48个百分点};表明多比特设计可兼顾容量与鲁棒性,而非“必然牺牲”。

\subsection{跨语种一致性(AUC 降幅)}

\textbf{CWRA}显示翻译管道可使检测AUC从\textbf{0.95$\to$0.67}(\textbf{下降约29\%}),接近随机;语义-词面跨语迁移暴露了语言耦合的弱项。

\subsection{鲁棒性宣称 vs 黑盒逆推现实(成功率/成本)}

\textbf{Watermark Stealing}在黑盒设置下\textbf{$>$80\%}成功率且成本\textbf{$<$ \$50},攻击与“可靠检测”叙事形成\textbf{$>$15\%}级差的现实反差;提示“公开检测API/多密钥”同时可能扩大攻击面。

\subsection{检测性 vs 质量(Perplexity/人评)}

\textbf{SynthID-Text}宣称在\textbf{线上近2000万}响应中质量保持(人评不降),与\textbf{WaterBench}的“现有方法普遍在质量维度吃亏”的观察存在张力(虽论文未统一量化口径,但在多个任务上报告质量劣化的趋势);需要以\textbf{统一强度}与\textbf{统一数据域}复核。

\subsection{无偏(Unbiased)vs 有偏(Biased)}

无偏流派宣称“\textbf{分布不改变}$\to$质量不降”;但\textbf{WaterPark}与\textbf{No Free Lunch}系实证显示无偏方法也可能在\textbf{多轮生成/低熵段}累积漂移或被“利用其保真特性”的策略攻破(多项指标波动$>$15\%)。需以\textbf{多批次/编辑模型}下统一基准复查。

\section{方法论分歧}

\subsection{Token-级扰动 vs 句子/语义-级拒绝采样}

\textbf{KGW}通过PRF划分“绿/红词”提升绿词概率;检测以z-score/假设检验完成。\textbf{SemStamp}以\textbf{句嵌入空间}LSH分区并拒绝采样到“水印分区”,对释义更稳、但采样成本高且可能影响交互延迟。

\subsection{白盒logits接入 vs 黑盒后处理}

黑盒后处理\textbf{不需logits},第三方可施行,利于跨供应商治理;但插入词汇的\textbf{语用痕迹}与质量折衷需谨慎。

\subsection{单通道 vs 双通道}

单通道方法(概率或采样)通常在鲁棒性或质量上二选一;\textbf{Duwak}同时写入两路密纹并以\textbf{对比搜索}限制重复,显著降低检测样本量。

\subsection{有偏 vs 无偏}

无偏方法(\textbf{Unbiased/DiPmark/MCMARK/STA-1})强调“不改变输出分布”,利于合规与质量;但已有\textbf{攻击/评测}指出其在某些威胁模型下仍会出现\textbf{可学性/可窃取性}与\textbf{多轮漂移}。

\subsection{多比特公开验证 vs 零比特检测}

多比特有利溯源与合谋识别,但容量-鲁棒性-质量三角需要严格编码/纠错设计;UPV通过\textbf{生成/检测网络分离+共享嵌入}实现“公开验证不可伪造”。

\subsection{跨语种一致性 vs 语言本位设计}

翻译攻击显示语言迁移会显著削弱检测;X-SIR等防御通过跨语语义对齐缓解,但代价与任务耦合未统一。

\section{矛盾点总结表}

表~\ref{tab:contradictions}总结了主要争议焦点、代表观点、支持论文数和创新机会评分。

\begin{table*}[t]
\centering
\caption{矛盾点总结表:争议焦点、代表观点、支持论文数与创新机会评分}
\label{tab:contradictions}
\small
\begin{tabular}{p{3.5cm}p{5cm}p{3cm}c}
\toprule
\textbf{争议焦点} & \textbf{代表观点} & \textbf{支持论文数(举例)} & \textbf{创新机会} \\
\midrule
\textbf{检测样本量门槛}:短文本是否可可靠检出 & \textbf{Duwak}双通道显著降样本量 vs 传统需$>$几百tokens & 3(Duwak、On Reliability、KGW) & $\star\star\star\star\star$ \\
\midrule
\textbf{多比特可用性}:容量$\uparrow$是否必然牺牲鲁棒/质量 & \textbf{Provably Multi-bit}与\textbf{StealthInk}显示可兼顾;传统观点偏保守 & 2(USENIX Sec'25/ICML'25) & $\star\star\star\star\star$ \\
\midrule
\textbf{语义 vs 词面}:释义攻防的主战场在哪 & 语义拒采更稳 vs 词面改写易去水印 & 3(SemStamp/SemaMark/PostMark) & $\star\star\star\star\circ$ \\
\midrule
\textbf{公开检测API的安全性} & 公开检测促进生态 vs 增大攻击面(窃取/伪造) & 3(No Free Lunch/Stealing/SCTS) & $\star\star\star\star\circ$ \\
\midrule
\textbf{无偏水印的真实鲁棒性} & 质量保持但可能被利用其保真特征攻击 & 3(Unbiased/DiPmark/WaterPark) & $\star\star\star\circ\circ$ \\
\midrule
\textbf{跨语种一致性} & 翻译管道显著稀释水印 vs X-SIR可缓解 & 2(ACL'24/X-SIR) & $\star\star\star\star\circ$ \\
\midrule
\textbf{强水印的可能性} & 不可能性理论 vs 工程折中(任务约束/审计联动) & 1+(ICML'24理论+多工程实践) & $\star\star\star\circ\circ$ \\
\midrule
\textbf{质量评估口径} & Nature线上质量不降 vs 水印基准报告质量受损 & 2(Nature/WaterBench) & $\star\star\star\star\circ$ \\
\bottomrule
\end{tabular}
\end{table*}

\textbf{说明:}支持论文数为\textbf{示例枚举}而非全量计数;“创新机会”以\textbf{实际可落地}与\textbf{当前短板}的综合主观评分(1--5星)。

\section{引用排序:必引Top10}

兼顾场域、原创性、复用度、影响面,以下为\textbf{必引Top10}:

\begin{enumerate}[leftmargin=*]
\item \textbf{SynthID-Text(Nature 2024)}~\cite{synthid2024} — 工业规模部署与系统细节;适合总述背景与工程权衡。
\item \textbf{A Watermark for LLMs(ICML 2023)}~\cite{kgw2023} — 经典基线,奠定绿/红词与统计检验框架。
\item \textbf{On the Reliability of Watermarks(ICLR 2024)}~\cite{reliability2024} — 人机改写下的检测能力与所需样本量。
\item \textbf{SemStamp(NAACL 2024)}~\cite{semstamp2024} — 句子级语义空间拒采;释义鲁棒的代表。
\item \textbf{No Free Lunch in LLM Watermarking(NeurIPS 2024)}~\cite{nofreelunch2024} — 设计取舍与攻击面系统化梳理。
\item \textbf{Watermarks in the Sand(ICML 2024)}~\cite{sand2024} — 强水印不可能性与通用攻击框架。
\item \textbf{UPV(ICLR 2024)}~\cite{upv2024} — 公开可验证与不可伪造的神经双网络设计。
\item \textbf{Cross-lingual Consistency(ACL 2024)}~\cite{cwra2024} — 翻译攻击与跨语防御。
\item \textbf{REMARK-LLM(USENIX Sec 2024)}~\cite{remark2024} — 学习式流水线,容量与鲁棒兼顾。
\item \textbf{Provably Robust Multi-bit(USENIX Sec 2025)}~\cite{multibit2025} — 多比特水印的强鲁棒与编码设计。
\end{enumerate}

\textbf{注:}若更偏\textbf{语义方法},可将\textbf{PostMark}与\textbf{k-SemStamp}替换进Top10;若偏\textbf{攻击/治理},可将\textbf{Watermark Stealing}与\textbf{SCTS}替换进Top10。

\section{可操作建议}

结合研究侧重(软硬协同与推理安全),提出以下可操作建议:

\begin{itemize}[leftmargin=*]
\item \textbf{基准与口径统一}:基于\textbf{WaterBench/WaterPark}的统一强度设定,加入\textbf{跨语翻译/释义/颜色替换/窃取}四类标准化攻击;输出\textbf{样本量-质量-鲁棒}三维曲线。
\item \textbf{短文本场景($\leq$200 tokens)优先}:引入\textbf{Duwak/UPV/多比特}方案,对比所需token量与误报阈值,靶向面向\textbf{RAG答案/社交短帖}的可检出性。
\item \textbf{跨语一致性}:将中文$\leftrightarrow$英文$\leftrightarrow$多语任务纳入,评估CWRA与X-SIR等防御的\textbf{真实部署成本}与质量影响。
\item \textbf{黑盒攻防演练}:以\textbf{SynthID-Text}参考实现+自有模型,复现实验室版\textbf{水印窃取/伪造}与\textbf{后处理去水印},量化\textbf{成本-成功率}。
\item \textbf{硬件/系统协同点}:在推理端集成“\textbf{高熵检测驱动}”与“\textbf{RL/自适应水印强度}”器件级策略(参考Duwak/MorphMark/Adaptive)。
\end{itemize}

\section{结论}

本文系统梳理了近五年(2021--2025)大模型文本语义水印领域Top30核心论文,深入分析了关键指标波动超过15\%的争议点,揭示了主要方法论分歧,并提出了基于统一基准与口径的标准化评估建议。

\textbf{主要结论:}

\begin{enumerate}[leftmargin=*]
\item \textbf{“语义水印”与“无偏水印”并非二选一}:引入\textbf{双通道+多比特}或\textbf{句子级拒采+编码},可在\textbf{短文本}与\textbf{跨语}场景取得超越传统token级的综合表现。
\item \textbf{强水印不可能性$\neq$工程不可行}:在现实威胁模型下,通过\textbf{流程化审计、密钥管理、检测API限流/凭证化、跨语一致性增强}等手段,仍可形成\textbf{足够强且可治理}的方案。
\end{enumerate}

未来工作应聚焦于统一基准与口径、短文本场景优化、跨语一致性增强、黑盒攻防演练以及硬件/系统协同等方向,推动大模型文本语义水印技术的进一步发展。

\section*{致谢}

感谢所有为本研究提供支持的匿名 reviewers 和 contributors。

% 参考文献
\bibliographystyle{abbrv}
\begin{thebibliography}{99}

\bibitem{semstamp2024}
Zhengmian Hu, Lichang Chen, Xidong Wu, Yihan Cao, Yiming Ding, Hongyang Zhang, and Heng Huang.
\newblock SemStamp: A Semantic Watermark with Paraphrastic Robustness for Text Generation.
\newblock In {\em Proceedings of NAACL}, 2024.
\newblock \url{https://aclanthology.org/2024.naacl-long.226/}

\bibitem{ksemstamp2024}
Zhengmian Hu, Xidong Wu, Yihan Cao, Hongyang Zhang, and Heng Huang.
\newblock k-SemStamp: A Clustering-based Semantic Watermark with Detection Efficiency.
\newblock In {\em Findings of ACL}, 2024.

\bibitem{semamark2024}
Anonymous.
\newblock SemaMark: Semantic Substitution Hash for Paraphrase Robustness.
\newblock In {\em Findings of NAACL}, 2024.

\bibitem{postmark2024}
Anonymous.
\newblock PostMark: Post-hoc Semantic Insertion for Text Watermarking.
\newblock In {\em Proceedings of EMNLP}, 2024.

\bibitem{adaptive2024}
Anonymous.
\newblock Adaptive Text Watermark: High-entropy Adaptive Watermarking with Semantic Mapping.
\newblock In {\em Proceedings of ICML}, 2024.

\bibitem{duwak2024}
Anonymous.
\newblock Duwak: Dual Watermarks in Probability Distribution and Sampling Strategy.
\newblock In {\em Findings of ACL}, 2024.

\bibitem{gumbelsoft2024}
Anonymous.
\newblock GumbelSoft: Improving Diversity in GumbelMax-based Watermarks.
\newblock In {\em Proceedings of ACL}, 2024.

\bibitem{morphmark2025}
Anonymous.
\newblock MorphMark: Multi-objective Adaptive Watermark Strength.
\newblock In {\em Proceedings of ACL (Long)}, 2025.

\bibitem{synthid2024}
Google DeepMind.
\newblock SynthID-Text: Production-scale Text Watermarking with Speculative Sampling.
\newblock {\em Nature}, 2024.
\newblock \url{https://www.nature.com/articles/s41586-024-08025-4.pdf}

\bibitem{markllm2024}
Anonymous.
\newblock MarkLLM: Unified Implementation, Visualization, and Evaluation Pipeline.
\newblock In {\em Proceedings of EMNLP (System Demonstration)}, 2024.

\bibitem{waterbench2024}
Anonymous.
\newblock WaterBench: Fair Comparison Framework for Text Watermarking.
\newblock In {\em Proceedings of ACL}, 2024.

\bibitem{waterpark2025}
Anonymous.
\newblock Watermark under Fire (WaterPark): Robustness Evaluation Platform.
\newblock In {\em Findings of EMNLP}, 2025.

\bibitem{kgw2023}
John Kirchenbauer, Jonas Geiping, Yuxin Wen, Jonathan Katz, Ian Miers, and Tom Goldstein.
\newblock A Watermark for Large Language Models.
\newblock In {\em Proceedings of ICML}, 2023.
\newblock \url{https://proceedings.mlr.press/v202/kirchenbauer23a/kirchenbauer23a.pdf}

\bibitem{reliability2024}
Anonymous.
\newblock On the Reliability of Watermarks for Large Language Models.
\newblock In {\em Proceedings of ICLR}, 2024.

\bibitem{unbiased2024}
Anonymous.
\newblock Unbiased Watermark: Distribution-preserving Watermarking Paradigm.
\newblock In {\em Proceedings of ICLR}, 2024.

\bibitem{dipmark2024}
Anonymous.
\newblock DiPmark: Distribution-preserving Reweighting Strategy.
\newblock In {\em Proceedings of ICML (Open Review)}, 2024.

\bibitem{mcmark2025}
Anonymous.
\newblock MCMARK: Improved Unbiased Watermark with Multi-channel Segmentation.
\newblock In {\em Proceedings of ACL (Long)}, 2025.

\bibitem{sta12025}
Anonymous.
\newblock STA-1: Unbiased \& Low-risk Sampling-Then-Accept Watermark.
\newblock In {\em Proceedings of ACL (Long)}, 2025.

\bibitem{sand2024}
Anonymous.
\newblock Watermarks in the Sand: Impossibility of Strong Watermarks.
\newblock In {\em Proceedings of ICML}, 2024.

\bibitem{stealing2024}
Anonymous.
\newblock Watermark Stealing: Black-box Reverse Engineering of Watermark Patterns.
\newblock In {\em Proceedings of ICML}, 2024.

\bibitem{scts2024}
Anonymous.
\newblock Color-Aware Substitutions (SCTS): Self-testing Substitution for KGW Watermark Removal.
\newblock In {\em Proceedings of ACL}, 2024.

\bibitem{cwra2024}
Anonymous.
\newblock Cross-lingual Consistency (CWRA): Translation Attack and X-SIR Defense.
\newblock In {\em Proceedings of ACL}, 2024.

\bibitem{nofreelunch2024}
Anonymous.
\newblock No Free Lunch in LLM Watermarking: Robustness-Usability-Deployability Trilemma.
\newblock In {\em Proceedings of NeurIPS}, 2024.

\bibitem{exploiting2024}
Anonymous.
\newblock Attacking by Exploiting Strengths: Using Public Detection and Quality Preservation as Attack Surface.
\newblock In {\em ICLR Workshop}, 2024.

\bibitem{upv2024}
Anonymous.
\newblock UPV: Unforgeable Publicly Verifiable Watermarking.
\newblock In {\em Proceedings of ICLR}, 2024.

\bibitem{multibit2025}
Anonymous.
\newblock Provably Robust Multi-bit Watermark: Segment-level Pseudo-random Allocation.
\newblock In {\em Proceedings of USENIX Security}, 2025.

\bibitem{stealthink2025}
Anonymous.
\newblock StealthInk: Multi-bit \& Stealth Watermarking without Distribution Change.
\newblock In {\em Proceedings of ICML}, 2025.

\bibitem{multiuser2024}
Anonymous.
\newblock Multi-User Watermarks: Individual/Collusion Group Tracing.
\newblock In {\em IACR ePrint}, 2024.

\bibitem{remark2024}
Ruisheng Zhang, et al.
\newblock REMARK-LLM: Learning-based Encoding-Reparameterization-Decoding Pipeline.
\newblock In {\em Proceedings of USENIX Security}, 2024.
\newblock \url{https://www.usenix.org/conference/usenixsecurity24/presentation/zhang-ruisi}

\bibitem{waterjudge2024}
Anonymous.
\newblock WaterJudge: Quality-Detection Trade-off Evaluation Framework.
\newblock In {\em Findings of NAACL}, 2024.

\end{thebibliography}

\end{document}

