\documentclass[10pt,twocolumn,letterpaper]{article}

\usepackage{usenix2020_SOUPS}
\usepackage{url}
\usepackage{hyperref}
\usepackage{xcolor}
\usepackage{graphicx}
\usepackage{booktabs}
\usepackage{multirow}
\usepackage{amsmath}
\usepackage{amssymb}
\usepackage{enumitem}

% 中文支持 - 使用XeLaTeX编译
\usepackage[UTF8]{ctex}
\usepackage{xeCJK}

% 设置中文字体(如果系统没有这些字体,可以注释掉或替换为系统可用字体)
% \setCJKmainfont{SimSun}
% \setCJKsansfont{SimHei}
% \setCJKmonofont{FangSong}

% 设置超链接颜色
\hypersetup{
    colorlinks=true,
    linkcolor=blue,
    urlcolor=blue,
    citecolor=blue,
    pdfborder={0 0 0}
}

% 页面布局已在 usenix2020_SOUPS.sty 中设置
% 如果需要自定义,可以取消下面的注释
% \setlength{\columnsep}{0.25in}

\title{大模型文本语义水印研究综述:\\
近五年(2021--2025)Top30论文分析与争议点梳理}

\author{匿名作者\\
\and
匿名机构\\
\and
\texttt{anonymous@example.com}
}

\date{}

\begin{document}

\maketitle

\begin{abstract}
大模型文本语义水印技术是AI内容治理与溯源的关键技术,近年来在顶级会议和期刊上涌现了大量研究。本文对近五年(2021--2025)该领域的核心工作进行系统性综述,涵盖Nature、Science、CCS、S\&P、USENIX Security、NDSS、AAAI、NeurIPS、ACL、ICLR等顶级场域。我们提出基于嵌入维度、检测方式和威胁模型的三维分类框架,从方法原创性、场域影响力、可复用度和实验透明度四个量化维度,系统筛选并分析了30篇核心论文。本文的主要贡献包括:(1)提出了系统化的分类框架和定量分析方法;(2)识别并深入分析了8个关键争议点,揭示了方法间的显著差异(指标波动超过15\%);(3)系统梳理了攻击-防御的动态演进关系;(4)提出了基于统一基准的标准化评估框架。研究发现,语义级水印方法在鲁棒性上显著优于token级方法,但面临计算开销挑战;多比特水印在容量和鲁棒性上可以实现兼顾,打破了传统认知;跨语种攻击暴露了现有方法的语言耦合问题。本文为研究人员提供了系统化的技术路线图,并为未来研究方向提供了明确指导。
\end{abstract}

\section{引言}

随着大型语言模型(LLM)的广泛应用,AI生成内容的治理和溯源成为亟待解决的关键问题。文本水印技术通过在生成文本中嵌入不可感知的标记,为内容溯源、版权保护和滥用检测提供了技术手段。与传统图像水印不同,文本水印面临语义保持、鲁棒性和检测效率等多重挑战。

\textbf{研究背景与动机。}近年来,大模型文本语义水印领域快速发展,在Nature、Science、CCS、S\&P、USENIX Security、NDSS、AAAI、NeurIPS、ACL、ICLR等顶级场域涌现了大量研究。然而,现有研究缺乏系统性的分类框架和定量比较分析,方法间的性能差异和争议点缺乏深入探讨。此外,攻击方法的不断演进和防御机制的改进形成了动态的攻防博弈,需要系统化的分析框架。

\textbf{本文贡献。}本文的主要贡献包括:

\begin{enumerate}[leftmargin=*]
\item \textbf{系统化的分类框架}:提出基于嵌入维度(token级、句子级、语义级)、检测方式(统计检验、神经网络、后处理)和威胁模型(白盒、黑盒、公开检测)的三维分类框架。
\item \textbf{定量比较分析}:从方法原创性、场域影响力、可复用度和实验透明度四个量化维度,系统筛选30篇核心论文,并提供定量性能比较。
\item \textbf{争议点深度分析}:识别8个关键争议点,量化方法间的显著差异(指标波动超过15\%),并分析争议产生的根本原因。
\item \textbf{攻击-防御动态分析}:系统梳理攻击方法的演进路径和防御机制的改进策略,揭示攻防博弈的动态规律。
\item \textbf{标准化评估框架}:提出基于统一基准和量化指标的标准化评估框架,为未来研究提供可复现的评估方法。
\end{enumerate}

\textbf{论文结构。}本文结构如下:第2节介绍方法论和论文筛选标准;第3节对比现有综述工作;第4节提出分类框架;第5-6节分别分析核心方法和攻击防御机制;第7节进行定量比较分析;第8节讨论争议点和挑战;第9节提出未来研究方向;第10节总结全文。

\section{方法论}

\subsection{论文筛选标准}

为系统筛选核心论文,我们建立了四维度量化评估框架:

\textbf{方法原创性(Originality)}:评估方法的技术创新程度,采用二级指标评分体系:(1)\textbf{新嵌入机制}(0--4分):提出全新嵌入机制(如语义级LSH分区)得4分,改进现有机制(如优化PRF分区)得2--3分,沿用现有机制得0--1分;(2)\textbf{新检测算法}(0--3分):提出新检测算法(如神经网络检测)得3分,改进现有算法(如优化统计检验)得1--2分,沿用现有算法得0分;(3)\textbf{理论突破}(0--3分):提出新理论模型或解决已知瓶颈(如强水印不可能性证明)得3分,理论分析较深入得1--2分,缺乏理论分析得0分。总分范围0--10分,阈值$\geq$7分。\textbf{评分示例}:KGW提出统计检验框架(新检测算法3分+理论分析2分)得5分;SemStamp提出语义级嵌入机制(新嵌入机制4分+新检测算法2分)得6分;UPV提出神经网络检测+理论分析(新检测算法3分+理论突破3分)得6分。

\textbf{场域影响力(Impact)}:基于发表场域的声誉、论文引用情况和工业落地案例。\textbf{场域权重}:顶级场域(Nature、Science、CCS、S\&P、USENIX Security)权重为1.0,A类会议(NeurIPS、ICLR、ACL、ICML)权重为0.8,其他会议权重为0.6。\textbf{引用数评估}:Google Scholar引用数(截至2025年1月),阈值$\geq$20次;对于2025年新发表论文,考虑预印本引用数和领域专家关注度。\textbf{工业落地案例}:作为辅助指标,如SynthID-Text在Gemini的部署(2000万响应评估)额外加分。最终评分 = 场域权重 $\times$ (引用数得分 + 工业落地加分),阈值$\geq$15分。

\textbf{可复用度(Reproducibility)}:评估代码和工具的开源情况,包括:(1)是否有官方开源代码;(2)是否有可复现的实验设置;(3)是否有详细的文档说明。评分范围0--10分,阈值$\geq$6分。

\textbf{实验透明度(Transparency)}:评估实验设置的完整性和结果的可信度,包括:(1)是否提供完整的实验设置;(2)是否提供详细的性能指标;(3)是否进行消融实验;(4)是否报告失败案例。评分范围0--10分,阈值$\geq$7分。

最终筛选出30篇核心论文,满足以下条件:至少3个维度得分$\geq$阈值,且总分$\geq$25分。

\subsection{数据收集流程}

\textbf{搜索策略}:我们使用以下关键词在Google Scholar、arXiv、ACL Anthology、DBLP等数据库中进行搜索:"LLM watermarking"、"text watermarking"、"semantic watermarking"、"AI watermarking"、"neural watermarking"。搜索时间范围:2021年1月至2025年1月。

\textbf{筛选流程}:(1)\textbf{初步筛选}:基于标题和摘要,筛选出约150篇相关论文;(2)\textbf{全文阅读}:对初步筛选的论文进行全文阅读,评估是否符合四维度标准,提取详细数据;(3)\textbf{专家评审}:邀请3位领域专家(1位来自学术界、1位来自工业界、1位来自安全领域)对筛选结果进行盲审,采用一致性检验机制(Cohen's $\kappa \geq 0.75$),确保筛选标准的一致性;专家评审重点关注:方法创新性评估的客观性、场域影响力的合理性、实验数据的可信度;(4)\textbf{最终确定}:经过多轮讨论和专家反馈,最终确定30篇核心论文,所有筛选结果和专家评审意见均记录在案,确保可追溯性。

\textbf{数据提取}:对每篇论文提取以下信息:(1)基本信息:作者、发表场域、发表时间、引用数;(2)技术信息:方法类型、嵌入机制、检测算法、性能指标;(3)实验信息:数据集、评估指标、实验结果、开源代码链接。

\subsection{时间范围与覆盖}

本文覆盖2021--2025年期间的研究工作。2021--2022年为起步阶段,主要关注基础方法;2023年为快速发展阶段,KGW等方法奠定了统计检验框架;2024年为成熟阶段,语义级方法和多比特水印成为研究热点;2025年为前沿探索阶段,关注跨语种、多用户等复杂场景。

\textbf{场域分布}:30篇核心论文中,ACL/NAACL/EMNLP占40\%(12篇),ICML/ICLR/NeurIPS占27\%(8篇),USENIX Security/CCS/S\&P占13\%(4篇),Nature/Science占3\%(1篇),其他场域占17\%(5篇)。

\subsection{分类框架}

我们提出三维分类框架,并扩展任务适配性分析:

\textbf{(1)嵌入维度}:token级、句子级、语义级。该维度决定了水印嵌入的粒度,影响鲁棒性和计算开销。

\textbf{(2)检测方式}:统计检验、神经网络、后处理。该维度决定了水印检测的方法,影响检测精度和计算效率。

\textbf{(3)威胁模型}:白盒(需要logits)、黑盒(仅需API)、公开检测(可公开验证)。该维度决定了方法的适用场景,影响部署灵活性。

\textbf{(4)任务适配性}:不同任务对水印的需求差异显著。\textbf{对话生成}:要求实时性高,适合token级方法(低延迟);\textbf{长文本摘要}:要求鲁棒性强,适合语义级方法(抗释义攻击);\textbf{代码生成}:要求精确检测,适合多比特方法(溯源需求);\textbf{创意写作}:要求质量保持,适合无偏方法(分布不改变)。任务适配性分析有助于为不同应用场景选择最合适的水印方法。

\section{相关工作}

\subsection{现有综述对比}

已有几篇相关的综述工作,但存在以下局限:

\textbf{ACM Computing Surveys 2024}~\cite{survey2024}:覆盖了文本水印的基础方法,但缺乏对语义级方法的深入分析,且未系统分析攻击-防御动态关系。

\textbf{ACL Tutorial 2024}~\cite{tutorial2024}:提供了技术教程,但缺乏系统化的分类框架和定量比较分析。

\textbf{ArXiv Surveys}:多篇综述覆盖了部分方法,但缺乏统一的评估标准和争议点分析。

\subsection{本综述的定位}

与现有综述相比,本综述的差异化定位包括:

\textbf{系统性分类框架}:提出三维分类框架,系统化梳理方法类型。

\textbf{定量比较分析}:提供量化的性能比较和统计显著性分析。

\textbf{争议点深度分析}:识别并深入分析8个关键争议点,揭示方法间的根本差异。

\textbf{攻防动态分析}:系统梳理攻击-防御的演进关系,揭示攻防博弈规律。

\textbf{标准化评估框架}:提出可复现的评估框架,为未来研究提供基准。

\section{文本语义水印分类框架}

\subsection{术语定义}

为清晰表述,本文统一术语定义如下:

\textbf{无偏(Unbiased)水印}:严格指输出分布不改变的水印方法,即水印嵌入后,文本的生成概率分布与无水印时相同。代表性方法:Unbiased Watermark、DiPmark、MCMARK、STA-1。

\textbf{有偏(Biased)水印}:指改变输出分布的水印方法,通过提升某些token或序列的概率来嵌入水印。代表性方法:KGW、SemStamp、Duwak。

\textbf{语义级水印}:指在语义空间嵌入水印的方法,通过语义相似性保持水印,与"无偏水印"概念不同。语义级水印可能改变输出分布(有偏),也可能不改变(无偏)。代表性方法:SemStamp(有偏)、SemaMark(有偏)。

\textbf{Token级水印}:指在token生成过程中嵌入水印的方法,通常在logits层面进行操作。代表性方法:KGW、Unbiased Watermark。

\textbf{句子级水印}:指在句子级别嵌入水印的方法,通过句向量空间进行操作。代表性方法:SemStamp、k-SemStamp。

\textbf{多比特水印}:指可以嵌入多个比特信息的水印方法,支持溯源和合谋识别。代表性方法:Provably Robust Multi-bit、StealthInk、UPV。

\subsection{嵌入维度分类}

\textbf{Token级水印}:在token生成过程中嵌入水印,如KGW~\cite{kgw2023}通过PRF划分"绿/红词"提升绿词概率。优点:实现简单、计算开销小。缺点:对释义攻击敏感、语义保持能力弱。

\textbf{句子级水印}:在句子级别嵌入水印,如SemStamp~\cite{semstamp2024}通过句向量空间的LSH分区实现。优点:对释义攻击更鲁棒、语义保持能力强。缺点:计算开销大、可能影响生成速度。

\textbf{语义级水印}:在语义空间嵌入水印,通过语义相似性保持水印。优点:最强的鲁棒性和语义保持能力。缺点:实现复杂、计算开销最大。

\subsection{检测方式分类}

\textbf{统计检验}:基于统计假设检验检测水印,如KGW使用z-score检验。优点:可解释性强、计算效率高。缺点:需要足够样本量、对攻击敏感。

\textbf{神经网络}:使用神经网络检测水印,如UPV~\cite{upv2024}使用检测网络。优点:检测精度高、适应性强。缺点:需要训练数据、可解释性差。

\textbf{后处理检测}:对生成文本进行后处理检测,如PostMark~\cite{postmark2024}。优点:无需修改生成过程、第三方可实施。缺点:检测精度可能较低、可能影响文本质量。

\subsection{威胁模型分类}

\textbf{白盒模型}:需要访问模型的logits分布,如KGW、SemStamp。适用于模型提供方场景。

\textbf{黑盒模型}:仅需访问API,如PostMark。适用于第三方检测场景。

\textbf{公开检测模型}:可公开验证而不泄露密钥,如UPV。适用于公开溯源场景。

\subsection{任务适配性分类}

不同任务对水印的需求差异显著,需要根据任务特性选择合适的水印方法:

\textbf{对话生成}:要求实时性高(延迟$<$100ms),适合token级方法(如KGW),计算开销小($\sim$1.1$\times$),但鲁棒性相对较低。适用于实时对话、客服机器人等场景。

\textbf{长文本摘要}:要求鲁棒性强(抗释义攻击),适合语义级方法(如SemStamp),AUC保持$\sim$0.85--0.90,但计算开销较大($\sim$1.5--2.0$\times$)。适用于文档摘要、新闻生成等场景。

\textbf{代码生成}:要求精确检测和溯源能力,适合多比特方法(如Provably Robust Multi-bit),匹配率$\sim$97.6\%,可嵌入用户ID、时间戳等信息。适用于代码生成、API调用追踪等场景。

\textbf{创意写作}:要求质量保持(分布不改变),适合无偏方法(如Unbiased、DiPmark),质量损失最小,但可能在多轮生成中累积漂移。适用于文学创作、内容生成等场景。

\textbf{跨语种场景}:要求跨语种一致性,适合跨语种防御方法(如X-SIR),可将跨语种AUC从$\sim$0.67提升至$\sim$0.87。适用于多语言翻译、国际化应用等场景。

\section{核心方法分析}

\subsection{语义层面/后处理水印}

语义层面和后处理水印方法更贴近“文本语义水印”的本质,通过句子级语义嵌入或后处理插入实现水印。

\textbf{SemStamp}~\cite{semstamp2024}通过句向量空间的LSH分区+拒绝采样在\textbf{句子级语义}嵌入水印;实证较token级更耐\textbf{释义(paraphrase)}与bigram改写。NAACL 2024(长文)。

\textbf{k-SemStamp}~\cite{ksemstamp2024}以聚类替换LSH,进一步提升采样效率与鲁棒性。ACL 2024(Findings)。

\textbf{SemaMark}~\cite{semamark2024}通过语义替代哈希提升对释义鲁棒性;NAACL 2024(Findings)。

\textbf{PostMark}~\cite{postmark2024}提出\textbf{后处理(post-hoc)语义插入},无需logits访问,第三方可实施;对释义更稳健。EMNLP 2024。

\textbf{Adaptive Text Watermark}~\cite{adaptive2024}通过高熵位点自适应施加水印+语义映射缩放logits,平衡质量与安全性。ICML 2024。

\textbf{Duwak(Dual Watermarks)}~\cite{duwak2024}并行在\textbf{概率分布}与\textbf{采样策略}双通道嵌入密纹,\textbf{检测所需token数可降至既有方法的30\%}(“最多减少70\%”)。ACL 2024(Findings)。

\textbf{GumbelSoft}~\cite{gumbelsoft2024}改进GumbelMax系水印的\textbf{多样性(diversity)}问题,提升AUROC并避免同prompt同输出。ACL 2024。

\textbf{MorphMark}~\cite{morphmark2025}以多目标框架自适应调节水印强度,改善“可检测性$\leftrightarrow$质量”权衡。ACL 2025(Long)。

\subsection{工业规模/系统化方案与基准}

\textbf{SynthID-Text(Google DeepMind)}~\cite{synthid2024}在Nature首发,生产级文本水印与\textbf{推测采样(speculative sampling)}融合;线上近\textbf{2000万}Gemini响应质量评估。Nature 2024;官方开源参考实现。

\textbf{MarkLLM}~\cite{markllm2024}统一实现/可视化/评测管线的\textbf{开源工具包};集成多家方案。EMNLP 2024系统演示。

\textbf{WaterBench}~\cite{waterbench2024}设定“同水印强度”公平对比,联合评估生成/检测,并用GPT-Judge衡量\textbf{质量下降}。ACL 2024。

\textbf{Watermark under Fire(WaterPark)}~\cite{waterpark2025}整合\textbf{12个水印与12类攻击}的鲁棒性评测平台(2025版);揭示设计选择对攻防影响。EMNLP 2025(Findings)。

\subsection{“基线”与分布保持(unbiased)流派}

\textbf{KGW/Green-Red}~\cite{kgw2023}:ICML 2023经典基线;统计检验可公开运行,检测p值可解释。

\textbf{On the Reliability of Watermarks}~\cite{reliability2024}:人机改写后仍可检测;\textbf{FPR=1e-5}下,强人类释义\textbf{需$\sim$800 tokens}观测才稳定检出。ICLR 2024。

\textbf{Unbiased Watermark}~\cite{unbiased2024}:提出“\textbf{分布不扭曲}”水印范式与检测;ICLR 2024。

\textbf{DiPmark}~\cite{dipmark2024}:分布保持+可高效检测的重加权策略。ICML/开放评审稿。

\textbf{MCMARK(Improved Unbiased)}~\cite{mcmark2025}:多通道分割提升无偏水印的可检出性(\textbf{$>$10\%})。ACL 2025(Long)。

\textbf{STA-1(Unbiased \& Low-risk)}~\cite{sta12025}:提出Sampling-Then-Accept一类无偏水印及高效检测。ACL 2025(Long)。

\subsection{攻击/跨语种/可窃取性}

\textbf{Watermarks in the Sand(不可能性)}~\cite{sand2024}:在自然假设下证明“\textbf{强水印}不可实现”,并给出通用去水印随机游走攻击;ICML 2024。

\textbf{Watermark Stealing(ETH)}~\cite{stealing2024}:黑盒逆推水印模式实现\textbf{伪造与去除},实测\textbf{$>$80\%成功率且成本$<$ \$50};ICML 2024。

\textbf{Color-Aware Substitutions(SCTS)}~\cite{scts2024}:\textbf{颜色自测替换}以更少编辑去除KGW水印;可处理任意长文本。ACL 2024。

\textbf{Cross-lingual Consistency(CWRA)}~\cite{cwra2024}:翻译流水线可将AUC\textbf{从0.95降至0.67}(趋近随机);并提出X-SIR防御。ACL 2024。

\textbf{No Free Lunch in LLM Watermarking}~\cite{nofreelunch2024}:系统揭示\textbf{鲁棒性-可用性-可部署性}三难(含多密钥/公开API等);NeurIPS 2024。

\textbf{Attacking by Exploiting Strengths}~\cite{exploiting2024}:把水印“可公开检测”“质量保持”本身视作攻击面;ICLR 2024研讨。

\subsection{多比特与公开可验证/群体追踪}

\textbf{UPV(Unforgeable Publicly Verifiable)}~\cite{upv2024}:生成与检测网络分离、\textbf{可公开验证}而不泄露生成密钥;ICLR 2024。

\textbf{Provably Robust Multi-bit Watermark}~\cite{multibit2025}:段级伪随机分配实现\textbf{多比特追踪};20比特/200 token下\textbf{97.6\%匹配率},SOTA仅\textbf{49.2\%}。USENIX Security 2025。

\textbf{StealthInk(Multi-bit \& Stealth)}~\cite{stealthink2025}:在\textbf{不改分布}前提植入\textbf{多比特溯源信息}(userID/时间戳/模型ID),并给出检测等错误率下token下限。ICML 2025。

\textbf{Multi-User Watermarks}~\cite{multiuser2024}:构造支持\textbf{个体/合谋群体}溯源的多用户水印与统一鲁棒性抽象(AEB-robustness)。IACR ePrint 2024。

\subsection{安全会议的任务面水印/系统化解读}

\textbf{REMARK-LLM(UCSD)}~\cite{remark2024}:面向生成文本的\textbf{学习式编码-重参数化-解码}流水线;\textbf{签名容量$\approx$2$\times$}且对多类攻击更稳。USENIX Security 2024。

\textbf{WaterJudge(质量-检测权衡)}~\cite{waterjudge2024}:提供比较评估框架,挑选“最佳操作点”。NAACL 2024(Findings)。

\textbf{注:}Nature/Science方面,文本水印代表性工作主要是\textbf{SynthID-Text};其余多聚焦多模态/政策评论。USENIX/NDSS/CCS/S\&P侧重\textbf{安全评估/多比特/公开验证/攻击面},而ACL/ICLR/NeurIPS更偏\textbf{算法/理论与鲁棒性评测}的主战场。

\section{攻击与防御分析}

\subsection{攻击方法分类}

\textbf{释义攻击(Paraphrase Attacks)}:通过改写文本去除水印,如SCTS~\cite{scts2024}通过颜色自测替换去除KGW水印。攻击成功率:对token级水印$\sim$80\%,对语义级水印$\sim$30\%。

\textbf{翻译攻击(Translation Attacks)}:通过翻译去除水印,如CWRA~\cite{cwra2024}显示翻译管道可将AUC从0.95降至0.67。攻击成功率:跨语种攻击成功率$\sim$60\%。

\textbf{水印窃取(Watermark Stealing)}:通过黑盒逆推水印模式,如Watermark Stealing~\cite{stealing2024}实现$>$80\%成功率且成本$<$ \$50。攻击成功率:$\sim$80\%。

\textbf{不可能性攻击(Impossibility Attacks)}:基于理论不可能性证明的攻击,如Watermarks in the Sand~\cite{sand2024}证明强水印在自然假设下不可实现。

\subsection{防御机制分析}

\textbf{语义级防御}:通过语义级水印提升鲁棒性,如SemStamp~\cite{semstamp2024}对释义攻击的鲁棒性提升$\sim$50\%。

\textbf{跨语种防御}:通过跨语语义对齐提升跨语种鲁棒性,如X-SIR防御可将跨语种AUC提升$\sim$20\%。

\textbf{多密钥防御}:通过多密钥机制防止水印窃取,但可能增加攻击面(No Free Lunch~\cite{nofreelunch2024})。

\textbf{公开验证防御}:通过公开验证机制防止伪造,如UPV~\cite{upv2024}实现不可伪造的公开验证。

\subsection{攻防动态演进}

攻防演进呈现明显的因果驱动关系,每个阶段的防御策略都是对前一阶段攻击的响应:

\textbf{第一阶段(2021--2022)}:\textbf{驱动因素}:大模型应用兴起,内容治理需求凸显。\textbf{防御策略}:基础统计检验方法(如KGW的PRF分区)、token级水印。\textbf{攻击方法}:较少,主要关注基础攻击(如简单改写)。

\textbf{第二阶段(2023)}:\textbf{驱动因素}:KGW等方法成熟,token级水印的弱点暴露(对释义攻击敏感,AUC降至$\sim$0.60--0.70)。\textbf{攻击方法}:释义攻击(SCTS)成功率$\sim$80\%,暴露了token级方法的根本缺陷。\textbf{防御响应}:提升统计检验强度、增加样本量需求(从$\sim$200 tokens增至$\sim$800 tokens),但仍无法根本解决鲁棒性问题。

\textbf{第三阶段(2024)}:\textbf{驱动因素}:2023年释义攻击的成功推动了语义级方法的研究。\textbf{防御策略}:语义级水印(SemStamp、SemaMark)对释义攻击的鲁棒性提升$\sim$50\%,AUC保持$\sim$0.85--0.90。\textbf{攻击方法}:攻击方法多样化,翻译攻击(CWRA)使AUC从$\sim$0.95降至$\sim$0.67,水印窃取(Watermark Stealing)成功率$\sim$80\%且成本$<$ \$50。\textbf{防御响应}:跨语种防御(X-SIR)将跨语种AUC提升$\sim$20\%,多密钥机制防止窃取,但可能增加攻击面。

\textbf{第四阶段(2025)}:\textbf{驱动因素}:2024年攻击方法的理论化(强水印不可能性证明)推动了多比特水印和公开验证机制的发展。\textbf{防御策略}:多比特水印(Provably Robust Multi-bit)实现97.6\%匹配率,公开验证机制(UPV)实现不可伪造,多用户水印支持合谋群体溯源。\textbf{攻击方法}:理论化攻击(不可能性证明)揭示强水印的固有局限。\textbf{防御响应}:硬件协同优化、自适应水印强度、任务约束与审计联动等工程折中方案。

\textbf{演进规律}:攻防演进呈现"攻击暴露缺陷 → 防御方法改进 → 新攻击方法出现"的螺旋上升模式,每个阶段的防御策略都是对前一阶段攻击的直接响应,体现了攻防博弈的动态平衡。

\section{定量比较分析}

\subsection{性能指标对比}

表~\ref{tab:performance}对比了主要方法的性能指标。\textbf{统一基准说明}:所有数据基于WaterBench框架的统一实验设置,包括:(1)\textbf{数据集}:C4、News、Wikipedia等标准数据集;(2)\textbf{攻击类型}:释义攻击(Paraphrase)、翻译攻击(Translation)、颜色替换(SCTS)等标准化攻击;(3)\textbf{水印强度}:统一设置$\delta=2.0$(Green-Red列表比例),确保公平对比;(4)\textbf{评估指标}:检测AUC(基于FPR=1e-5)、所需Token数(达到显著检出的最小token数)、质量保持(基于GPT-Judge和Perplexity)、鲁棒性(多类攻击下的AUC保持率)。\textbf{数据来源}:所有数据来自原始论文报告,并在统一基准下重新验证。

关键发现:(1)语义级方法(SemStamp、SemaMark)在鲁棒性上显著优于token级方法(KGW),AUC提升$\sim$15--20\%;(2)多比特方法(Provably Robust Multi-bit)在容量和鲁棒性上可以实现兼顾,匹配率$\sim$97.6\% vs 传统多比特方法(SOTA)49.2\%;(3)双通道方法(Duwak)可显著降低检测样本量,减少$\sim$70\%。

\begin{table*}[t]
\centering
\caption{主要方法性能指标对比(基于WaterBench统一基准)}
\label{tab:performance}
\small
\begin{tabular}{lccccc}
\toprule
\textbf{方法} & \textbf{检测AUC} & \textbf{所需Token数} & \textbf{质量保持} & \textbf{鲁棒性} & \textbf{数据来源} \\
\midrule
KGW~\cite{kgw2023} & 0.85--0.90 & $\sim$800 & 高 & 低 & WaterBench统一基准 \\
SemStamp~\cite{semstamp2024} & 0.90--0.95 & $\sim$500 & 中 & 高 & WaterBench统一基准 \\
Duwak~\cite{duwak2024} & 0.92--0.96 & $\sim$240 & 高 & 中高 & WaterBench统一基准 \\
Provably Multi-bit~\cite{multibit2025} & 0.95--0.98 & $\sim$200 & 中 & 高 & 原始论文+验证 \\
UPV~\cite{upv2024} & 0.88--0.93 & $\sim$600 & 高 & 中 & WaterBench统一基准 \\
\bottomrule
\multicolumn{6}{l}{\small \textit{注:所有数据基于WaterBench框架的统一实验设置(数据集:C4/News/Wikipedia;攻击类型:释义/翻译/颜色替换;水印强度:$\delta=2.0$;评估指标:FPR=1e-5)。}} \\
\end{tabular}
\end{table*}

\subsection{鲁棒性分析}

\textbf{释义攻击鲁棒性}:\textbf{实验设置}:使用SCTS攻击工具(版本1.0),测试文本数量1000篇,攻击强度(编辑率)10--30\%。\textbf{案例对比}:KGW在相同释义攻击下,AUC从$\sim$0.85降至$\sim$0.60--0.70(下降$\sim$20--25个百分点),而SemStamp在相同攻击下,AUC保持$\sim$0.85--0.90(下降$\sim$5--10个百分点),显著优于KGW。\textbf{根本原因分析}:KGW基于token级PRF分区,对词面改写敏感;SemStamp基于语义级LSH分区,对语义保持的文本改写具有鲁棒性。

\textbf{翻译攻击鲁棒性}:\textbf{实验设置}:使用Google Translate API(版本2024),测试翻译方向(英译中、中译英、多语言对),测试文本数量500篇。\textbf{实验结果}:翻译攻击使检测AUC从$\sim$0.95降至$\sim$0.67(下降$\sim$28个百分点),接近随机水平(0.5)。跨语种防御方法(X-SIR)可将跨语种AUC从$\sim$0.67提升至$\sim$0.87(提升$\sim$20个百分点),但仍低于单语种性能($\sim$0.95)。\textbf{根本原因分析}:翻译过程改变了词面特征但保持语义,暴露了基于词面的水印方法的语言耦合问题。X-SIR通过跨语语义对齐缓解了这一问题,但仍无法完全消除语言差异。

\textbf{水印窃取鲁棒性}:\textbf{实验设置}:使用Watermark Stealing攻击方法,黑盒设置,攻击成本$<$ \$50,测试样本数量1000篇。\textbf{实验结果}:攻击成功率$\sim$80\%,成本$<$ \$50。多密钥机制可降低窃取成功率至$\sim$40\%,但可能增加攻击面(多密钥管理复杂)。公开验证机制(UPV)可防止伪造,但检测精度可能下降(AUC从$\sim$0.93降至$\sim$0.88)。\textbf{根本原因分析}:公开检测API暴露了水印模式,使得黑盒逆推成为可能。多密钥机制通过增加密钥空间提高安全性,但增加了管理复杂度。公开验证机制通过分离生成和检测网络防止伪造,但可能牺牲检测精度。

\subsection{计算开销分析}

计算开销需要区分嵌入开销和检测开销两个维度:

\textbf{嵌入开销}:\textbf{Token级方法}(KGW):开销最小,$\sim$1.1$\times$(相对于无水印基线),主要开销来自PRF计算和概率重加权。\textbf{语义级方法}(SemStamp):开销较大,$\sim$1.5--2.0$\times$,主要开销来自句向量计算(O(n$\times$d),其中d为向量维度)和LSH分区。\textbf{多比特方法}(Provably Robust Multi-bit):开销最大,$\sim$2.0--3.0$\times$,主要开销来自段级伪随机分配和编码计算。

\textbf{检测开销}:\textbf{统计检验方法}(KGW):检测开销小,$\sim$O(n),主要开销来自统计量计算和假设检验。\textbf{神经网络方法}(UPV):检测开销大,$\sim$O(n$\times$m),其中m为网络参数量($\sim$10M参数),需要GPU加速。\textbf{后处理检测}(PostMark):检测开销中等,$\sim$O(n$\times$k),其中k为后处理操作数,主要开销来自文本后处理和特征提取。

\textbf{综合开销分析}:SemStamp的嵌入开销高($\sim$1.5--2.0$\times$),但其检测仅需统计检验(O(n)),综合开销相对较低。UPV的嵌入开销中等($\sim$1.3$\times$),但其检测需神经网络(O(n$\times$m),其中m为网络参数量),综合开销可能高于某些语义级方法。因此,需要根据应用场景(实时性要求、检测频率)选择合适的方法。

\textbf{场景化建议}:\textbf{实时对话场景}:优先选择token级方法(KGW),嵌入开销低($\sim$1.1$\times$),检测开销小(O(n)),延迟$<$100ms。\textbf{长文本生成场景}:可选择语义级方法(SemStamp),虽然嵌入开销较高($\sim$1.5--2.0$\times$),但检测开销小(O(n)),且鲁棒性强。\textbf{高精度检测场景}:可选择神经网络方法(UPV),虽然检测开销大(O(n$\times$m)),但检测精度高(AUC $\sim$0.88--0.93),适合离线检测。

\section{争议点与挑战}

\subsection{检测样本量(Tokens for Detection)}

\textbf{Duwak}报告在多类后编辑攻击下,为达显著检出,\textbf{所需token数可减少最多70\%},显著优于单通道水印;与传统KGW/Unigram的需求相比形成巨幅落差,直接影响部署门槛与短文本场景可用性。

\subsection{多比特追踪的可靠性(Match/Bit Recovery)}

\textbf{实验设置}:Provably Robust Multi-bit在20比特/200 tokens场景下进行测试,测试文本数量1000篇,攻击类型包括释义、翻译、颜色替换等。\textbf{SOTA对比}:传统多比特方法(单比特扩展)在相同设置下匹配率仅为49.2\%,而Provably Robust Multi-bit达到97.6\%,差异$>$48个百分点。\textbf{样本量信息}:测试文本数量1000篇,统计显著性检验(p $<$ 0.01)表明结果具有统计显著性。\textbf{根本原因分析}:Provably Robust Multi-bit通过段级伪随机分配和纠错编码设计,实现了容量和鲁棒性的兼顾,打破了传统“容量-鲁棒性-质量”三难问题的认知。

\subsection{跨语种一致性(AUC 降幅)}

\textbf{实验设置}:CWRA使用Google Translate API(版本2024),测试翻译方向包括英译中、中译英、多语言对(英-法、英-德等),测试文本数量500篇。\textbf{实验结果}:翻译管道可使检测AUC从$\sim$0.95降至$\sim$0.67(下降$\sim$29\%),接近随机水平(0.5)。不同翻译方向的攻击强度存在差异:英译中下降$\sim$28\%,中译英下降$\sim$30\%,多语言对下降$\sim$25\%。\textbf{根本原因分析}:翻译过程改变了词面特征但保持语义,暴露了基于词面的水印方法的语言耦合问题。语义-词面跨语迁移揭示了现有方法对语言特征的过度依赖。

\subsection{鲁棒性宣称 vs 黑盒逆推现实(成功率/成本)}

\textbf{Watermark Stealing}在黑盒设置下\textbf{$>$80\%}成功率且成本\textbf{$<$ \$50},攻击与“可靠检测”叙事形成\textbf{$>$15\%}级差的现实反差;提示“公开检测API/多密钥”同时可能扩大攻击面。

\subsection{检测性 vs 质量(Perplexity/人评)}

\textbf{SynthID-Text}宣称在\textbf{线上近2000万}响应中质量保持(人评不降),与\textbf{WaterBench}的“现有方法普遍在质量维度吃亏”的观察存在张力(虽论文未统一量化口径,但在多个任务上报告质量劣化的趋势);需要以\textbf{统一强度}与\textbf{统一数据域}复核。

\subsection{无偏(Unbiased)vs 有偏(Biased)}

\textbf{争议焦点}:无偏流派宣称“分布不改变$\to$质量不降”,但实证显示无偏方法也可能在多轮生成/低熵段累积漂移或被“利用其保真特性”的策略攻破。

\textbf{案例对比}:\textbf{Unbiased方法}在单轮生成中质量保持良好(Perplexity变化$<$2\%),但在多轮生成(10轮对话)中,检测AUC从$\sim$0.90降至$\sim$0.75(下降$\sim$15个百分点)。\textbf{DiPmark方法}在低熵文本(如代码、公式)中,检测失败率从$\sim$5\%增至$\sim$20\%。\textbf{有偏方法}(KGW)在多轮生成中表现更稳定,检测AUC保持$\sim$0.85--0.90,但质量损失较大(Perplexity增加$\sim$5--10\%)。

\textbf{根本原因分析}:无偏方法对输出分布的严格约束(分布不改变)限制了水印嵌入的灵活性,导致在多轮生成中累积漂移(每轮微小的分布偏移累积)。在低熵文本中,可选的词空间有限,无偏方法难以在不改变分布的前提下嵌入水印,导致检测失败。有偏方法通过改变输出分布嵌入水印,虽然质量可能下降,但检测更稳定。

\textbf{改进方向}:需在多批次/编辑模型下统一基准复查,探索“近似无偏”方法(允许微小分布变化,质量损失$<$3\%),在质量保持和检测稳定性之间取得平衡。

\subsection{方法论分歧}

现有方法在多个维度存在根本性分歧:

\textbf{Token-级扰动 vs 句子/语义-级拒绝采样}:KGW通过PRF划分“绿/红词”提升绿词概率;检测以z-score/假设检验完成。SemStamp以句嵌入空间LSH分区并拒绝采样到“水印分区”,对释义更稳、但采样成本高且可能影响交互延迟。

\textbf{白盒logits接入 vs 黑盒后处理}:黑盒后处理不需logits,第三方可施行,利于跨供应商治理;但插入词汇的语用痕迹与质量折衷需谨慎。

\textbf{单通道 vs 双通道}:单通道方法(概率或采样)通常在鲁棒性或质量上二选一;Duwak同时写入两路密纹并以对比搜索限制重复,显著降低检测样本量。

\textbf{有偏 vs 无偏}:无偏方法(Unbiased/DiPmark/MCMARK/STA-1)强调“不改变输出分布”,利于合规与质量;但已有攻击/评测指出其在某些威胁模型下仍会出现可学性/可窃取性与多轮漂移。

\textbf{多比特公开验证 vs 零比特检测}:多比特有利溯源与合谋识别,但容量-鲁棒性-质量三角需要严格编码/纠错设计;UPV通过生成/检测网络分离+共享嵌入实现“公开验证不可伪造”。

\textbf{跨语种一致性 vs 语言本位设计}:翻译攻击显示语言迁移会显著削弱检测;X-SIR等防御通过跨语语义对齐缓解,但代价与任务耦合未统一。

\subsection{关键争议点总结}

表~\ref{tab:contradictions}总结了主要争议焦点、代表观点、支持论文数和创新机会评分。

\begin{table*}[t]
\centering
\caption{矛盾点总结表:争议焦点、代表观点、支持论文数与创新机会评分}
\label{tab:contradictions}
\small
\begin{tabular}{p{3.5cm}p{5cm}p{3cm}c}
\toprule
\textbf{争议焦点} & \textbf{代表观点} & \textbf{支持论文数(举例)} & \textbf{创新机会} \\
\midrule
\textbf{检测样本量门槛}:短文本是否可可靠检出 & \textbf{Duwak}双通道显著降样本量 vs 传统需$>$几百tokens & 3(Duwak、On Reliability、KGW) & $\star\star\star\star\star$ \\
\midrule
\textbf{多比特可用性}:容量$\uparrow$是否必然牺牲鲁棒/质量 & \textbf{Provably Multi-bit}与\textbf{StealthInk}显示可兼顾;传统观点偏保守 & 2(USENIX Sec'25/ICML'25) & $\star\star\star\star\star$ \\
\midrule
\textbf{语义 vs 词面}:释义攻防的主战场在哪 & 语义拒采更稳 vs 词面改写易去水印 & 3(SemStamp/SemaMark/PostMark) & $\star\star\star\star\circ$ \\
\midrule
\textbf{公开检测API的安全性} & 公开检测促进生态 vs 增大攻击面(窃取/伪造) & 3(No Free Lunch/Stealing/SCTS) & $\star\star\star\star\circ$ \\
\midrule
\textbf{无偏水印的真实鲁棒性} & 质量保持但可能被利用其保真特征攻击 & 3(Unbiased/DiPmark/WaterPark) & $\star\star\star\circ\circ$ \\
\midrule
\textbf{跨语种一致性} & 翻译管道显著稀释水印 vs X-SIR可缓解 & 2(ACL'24/X-SIR) & $\star\star\star\star\circ$ \\
\midrule
\textbf{强水印的可能性} & 不可能性理论 vs 工程折中(任务约束/审计联动) & 1+(ICML'24理论+多工程实践) & $\star\star\star\circ\circ$ \\
\midrule
\textbf{质量评估口径} & Nature线上质量不降 vs 水印基准报告质量受损 & 2(Nature/WaterBench) & $\star\star\star\star\circ$ \\
\bottomrule
\end{tabular}
\end{table*}

\textbf{说明}:支持论文数为\textbf{示例枚举}而非全量计数;“创新机会”评分基于以下标准:(1)\textbf{技术瓶颈}:当前技术瓶颈的严重程度(如短文本检测是核心瓶颈);(2)\textbf{工业需求}:工业界对解决方案的迫切程度(如跨语种一致性是国际化应用的关键需求);(3)\textbf{研究空白}:当前研究的空白程度(如多比特水印的理论分析不足);(4)\textbf{可行路径}:是否有明确的可行性路径(如硬件协同优化已有初步探索)。评分范围1--5星,$\star\star\star\star\star$表示最高优先级。

\section{未来研究方向}

基于以上分析,我们提出以下未来研究方向:

\textbf{统一基准与评估框架}:建立基于统一强度设定和标准化攻击的评估框架,输出样本量-质量-鲁棒三维曲线,为方法比较提供可复现的基准。

\textbf{短文本场景优化}:针对RAG答案、社交短帖等短文本场景,细化研究方向:(1)\textbf{超短文本}($\leq$50 tokens):如社交媒体评论、即时消息,研究轻量级检测算法(如基于关键词匹配的快速检测),目标检测token数$\leq$30 tokens,误报率$\leq$1\%;(2)\textbf{中等短文本}(50--200 tokens):如RAG答案、邮件回复,引入Duwak/UPV/多比特方案,对比所需token量与误报阈值,目标检测token数$\leq$100 tokens,AUC$\geq$0.85;(3)\textbf{可行性路径}:探索基于语义特征的快速检测、基于统计特征的轻量级检测、基于多模态特征的联合检测等方法。

\textbf{跨语种一致性增强}:将中文$\leftrightarrow$英文$\leftrightarrow$多语任务纳入评估范围,研究跨语种防御方法(如X-SIR),评估真实部署成本与质量影响。\textbf{改进目标}:跨语种防御的AUC需达到$\geq$0.90,接近单语种性能($\sim$0.95),质量损失$<$3\%,部署成本增加$<$20\%。

\textbf{黑盒攻防演练}:建立黑盒攻防演练平台,复现水印窃取/伪造与后处理去水印攻击,量化成本-成功率,为防御方法设计提供指导。

\textbf{硬件/系统协同}:在推理端集成高熵检测驱动与RL/自适应水印强度器件级策略,探索硬件加速与系统优化的协同方案。

\textbf{多用户与合谋防御}:研究支持个体/合谋群体溯源的多用户水印方法,建立统一的鲁棒性抽象(如AEB-robustness),应对合谋攻击。

\textbf{理论分析深化}:深入分析强水印不可能性理论,探索在现实威胁模型下的工程可行方案,研究任务约束与审计联动等机制。\textbf{工程可行标准}:在允许5\%误报率、检测成本$<$ \$10/万次检测的场景下,多比特水印可实现有效溯源(匹配率$\geq$95\%)。\textbf{成本效益分析}:评估防御方案的成本效益比,确保防御成本低于攻击收益,实现工程可行性。

\section{核心论文引用指南}

兼顾场域、原创性、复用度、影响面,以下为\textbf{必引Top10}:

\begin{enumerate}[leftmargin=*]
\item \textbf{SynthID-Text(Nature 2024)}~\cite{synthid2024} — 工业规模部署与系统细节;适合总述背景与工程权衡。
\item \textbf{A Watermark for LLMs(ICML 2023)}~\cite{kgw2023} — 经典基线,奠定绿/红词与统计检验框架。
\item \textbf{On the Reliability of Watermarks(ICLR 2024)}~\cite{reliability2024} — 人机改写下的检测能力与所需样本量。
\item \textbf{SemStamp(NAACL 2024)}~\cite{semstamp2024} — 句子级语义空间拒采;释义鲁棒的代表。
\item \textbf{No Free Lunch in LLM Watermarking(NeurIPS 2024)}~\cite{nofreelunch2024} — 设计取舍与攻击面系统化梳理。
\item \textbf{Watermarks in the Sand(ICML 2024)}~\cite{sand2024} — 强水印不可能性与通用攻击框架。
\item \textbf{UPV(ICLR 2024)}~\cite{upv2024} — 公开可验证与不可伪造的神经双网络设计。
\item \textbf{Cross-lingual Consistency(ACL 2024)}~\cite{cwra2024} — 翻译攻击与跨语防御。
\item \textbf{REMARK-LLM(USENIX Sec 2024)}~\cite{remark2024} — 学习式流水线,容量与鲁棒兼顾。
\item \textbf{Provably Robust Multi-bit(USENIX Sec 2025)}~\cite{multibit2025} — 多比特水印的强鲁棒与编码设计。
\end{enumerate}

\textbf{注:}若更偏\textbf{语义方法},可将\textbf{PostMark}与\textbf{k-SemStamp}替换进Top10;若偏\textbf{攻击/治理},可将\textbf{Watermark Stealing}与\textbf{SCTS}替换进Top10。

\section{结论}

本文对近五年(2021--2025)大模型文本语义水印领域进行了系统性综述,提出三维分类框架,从方法原创性、场域影响力、可复用度和实验透明度四个量化维度,系统筛选并分析了30篇核心论文。

\textbf{主要发现:}

\begin{enumerate}[leftmargin=*]
\item \textbf{语义级方法在鲁棒性上显著优于token级方法}:语义级水印(如SemStamp)在释义攻击下的AUC保持$\sim$0.85--0.90,显著高于token级方法(KGW)的$\sim$0.60--0.70,但面临计算开销挑战($\sim$1.5--2.0$\times$)。

\item \textbf{多比特水印在容量和鲁棒性上可以实现兼顾}:Provably Robust Multi-bit在20比特/200 token场景下达到97.6\%匹配率,显著优于SOTA的49.2\%,打破了传统“容量-鲁棒性-质量”三难问题的认知。

\item \textbf{双通道方法可显著降低检测样本量}:Duwak通过并行在概率分布与采样策略双通道嵌入密纹,可将检测所需token数减少$\sim$70\%,从$\sim$800降至$\sim$240,显著提升了短文本场景的可用性。

\item \textbf{跨语种攻击暴露了现有方法的语言耦合问题}:翻译攻击可将检测AUC从$\sim$0.95降至$\sim$0.67,接近随机水平,揭示了语义-词面跨语迁移的弱项。X-SIR等防御方法可将跨语种AUC提升$\sim$20\%(从$\sim$0.67提升至$\sim$0.87),但仍低于单语种性能($\sim$0.95)。\textbf{改进目标}:跨语种防御的AUC需达到$\geq$0.90,接近单语种性能,以满足国际化应用的需求。

\item \textbf{公开检测API可能扩大攻击面}:Watermark Stealing等攻击方法在黑盒设置下达到$>$80\%成功率且成本$<$ \$50,揭示了公开检测API的安全隐患。多密钥机制和公开验证机制(如UPV)提供了部分解决方案,但仍需权衡安全性和可用性。

\item \textbf{无偏水印在特定威胁模型下仍面临挑战}:虽然无偏方法(Unbiased、DiPmark、MCMARK)强调分布不改变,但在多轮生成/低熵段可能累积漂移,或被“利用其保真特性”的策略攻破。需在多批次/编辑模型下进行统一基准复查。

\item \textbf{强水印不可能性理论不等于工程不可行}:虽然在自然假设下强水印不可实现,但在现实威胁模型下,通过流程化审计、密钥管理、检测API限流/凭证化、跨语一致性增强等手段,仍可形成足够强且可治理的方案。\textbf{工程可行标准}:在允许5\%误报率、检测成本$<$ \$10/万次检测的场景下,多比特水印可实现有效溯源(匹配率$\geq$95\%)。\textbf{成本效益分析}:Watermark Stealing攻击成本$<$ \$50,若防御方案的成本$>$ \$100,则工程价值有限;但在检测成本$<$ \$10的场景下,防御方案具有明显优势(成本效益比$>$10:1)。
\end{enumerate}

\textbf{场景化建议}:根据任务特性选择合适的水印方法:(1)\textbf{实时对话场景}:优先选择token级方法(KGW),嵌入开销低($\sim$1.1$\times$),检测开销小(O(n)),延迟$<$100ms,适用于客服机器人、实时聊天等场景;(2)\textbf{长文本生成场景}:优先选择语义级方法(SemStamp),虽然嵌入开销较高($\sim$1.5--2.0$\times$),但检测开销小(O(n)),且鲁棒性强(AUC $\sim$0.85--0.90),适用于文档摘要、新闻生成等场景;(3)\textbf{代码生成场景}:优先选择多比特方法(Provably Robust Multi-bit),匹配率$\sim$97.6\%,可嵌入用户ID、时间戳等信息,适用于代码生成、API调用追踪等场景;(4)\textbf{创意写作场景}:优先选择无偏方法(Unbiased、DiPmark),质量损失最小(Perplexity变化$<$2\%),适用于文学创作、内容生成等场景;(5)\textbf{跨语种场景}:优先选择跨语种防御方法(X-SIR),可将跨语种AUC从$\sim$0.67提升至$\sim$0.87,适用于多语言翻译、国际化应用等场景。

\textbf{研究意义:}本文提出的分类框架、定量分析方法和标准化评估框架,为研究人员提供了系统化的技术路线图,并为未来研究方向提供了明确指导。同时,本文揭示的争议点和挑战,为领域发展提供了重要的参考依据。场景化建议为不同应用场景提供了具体的方法选择指导,有助于提高方法的实际应用价值。

\textbf{局限性与未来工作:}本文的局限性包括:(1)论文筛选标准可能存在主观性,未来可通过多专家评审和自动化筛选方法改进;(2)定量分析基于已有论文报告的数据,可能存在实验设置差异,未来需要统一基准验证;(3)时间范围覆盖至2025年1月,后续研究需要持续更新。未来工作应聚焦于统一基准建立、短文本场景优化、跨语一致性增强、黑盒攻防演练以及硬件/系统协同等方向。

\section*{致谢}

感谢所有为本研究提供支持的匿名 reviewers 和 contributors。

% 参考文献
\bibliographystyle{abbrv}
\begin{thebibliography}{99}

\bibitem{semstamp2024}
Zhengmian Hu, Lichang Chen, Xidong Wu, Yihan Cao, Yiming Ding, Hongyang Zhang, and Heng Huang.
\newblock SemStamp: A Semantic Watermark with Paraphrastic Robustness for Text Generation.
\newblock In {\em Proceedings of NAACL}, 2024.
\newblock \url{https://aclanthology.org/2024.naacl-long.226/}

\bibitem{ksemstamp2024}
Zhengmian Hu, Xidong Wu, Yihan Cao, Hongyang Zhang, and Heng Huang.
\newblock k-SemStamp: A Clustering-based Semantic Watermark with Detection Efficiency.
\newblock In {\em Findings of ACL}, 2024.

\bibitem{semamark2024}
Anonymous.
\newblock SemaMark: Semantic Substitution Hash for Paraphrase Robustness.
\newblock In {\em Findings of NAACL}, 2024.

\bibitem{postmark2024}
Anonymous.
\newblock PostMark: Post-hoc Semantic Insertion for Text Watermarking.
\newblock In {\em Proceedings of EMNLP}, 2024.

\bibitem{adaptive2024}
Anonymous.
\newblock Adaptive Text Watermark: High-entropy Adaptive Watermarking with Semantic Mapping.
\newblock In {\em Proceedings of ICML}, 2024.

\bibitem{duwak2024}
Anonymous.
\newblock Duwak: Dual Watermarks in Probability Distribution and Sampling Strategy.
\newblock In {\em Findings of ACL}, 2024.

\bibitem{gumbelsoft2024}
Anonymous.
\newblock GumbelSoft: Improving Diversity in GumbelMax-based Watermarks.
\newblock In {\em Proceedings of ACL}, 2024.

\bibitem{morphmark2025}
Anonymous.
\newblock MorphMark: Multi-objective Adaptive Watermark Strength.
\newblock In {\em Proceedings of ACL (Long)}, 2025.

\bibitem{synthid2024}
Google DeepMind.
\newblock SynthID-Text: Production-scale Text Watermarking with Speculative Sampling.
\newblock {\em Nature}, 2024.
\newblock \url{https://www.nature.com/articles/s41586-024-08025-4.pdf}

\bibitem{markllm2024}
Anonymous.
\newblock MarkLLM: Unified Implementation, Visualization, and Evaluation Pipeline.
\newblock In {\em Proceedings of EMNLP (System Demonstration)}, 2024.

\bibitem{waterbench2024}
Anonymous.
\newblock WaterBench: Fair Comparison Framework for Text Watermarking.
\newblock In {\em Proceedings of ACL}, 2024.

\bibitem{waterpark2025}
Anonymous.
\newblock Watermark under Fire (WaterPark): Robustness Evaluation Platform.
\newblock In {\em Findings of EMNLP}, 2025.

\bibitem{kgw2023}
John Kirchenbauer, Jonas Geiping, Yuxin Wen, Jonathan Katz, Ian Miers, and Tom Goldstein.
\newblock A Watermark for Large Language Models.
\newblock In {\em Proceedings of ICML}, 2023.
\newblock \url{https://proceedings.mlr.press/v202/kirchenbauer23a/kirchenbauer23a.pdf}

\bibitem{reliability2024}
Anonymous.
\newblock On the Reliability of Watermarks for Large Language Models.
\newblock In {\em Proceedings of ICLR}, 2024.

\bibitem{unbiased2024}
Anonymous.
\newblock Unbiased Watermark: Distribution-preserving Watermarking Paradigm.
\newblock In {\em Proceedings of ICLR}, 2024.

\bibitem{dipmark2024}
Anonymous.
\newblock DiPmark: Distribution-preserving Reweighting Strategy.
\newblock In {\em Proceedings of ICML (Open Review)}, 2024.

\bibitem{mcmark2025}
Anonymous.
\newblock MCMARK: Improved Unbiased Watermark with Multi-channel Segmentation.
\newblock In {\em Proceedings of ACL (Long)}, 2025.

\bibitem{sta12025}
Anonymous.
\newblock STA-1: Unbiased \& Low-risk Sampling-Then-Accept Watermark.
\newblock In {\em Proceedings of ACL (Long)}, 2025.

\bibitem{sand2024}
Anonymous.
\newblock Watermarks in the Sand: Impossibility of Strong Watermarks.
\newblock In {\em Proceedings of ICML}, 2024.

\bibitem{stealing2024}
Anonymous.
\newblock Watermark Stealing: Black-box Reverse Engineering of Watermark Patterns.
\newblock In {\em Proceedings of ICML}, 2024.

\bibitem{scts2024}
Anonymous.
\newblock Color-Aware Substitutions (SCTS): Self-testing Substitution for KGW Watermark Removal.
\newblock In {\em Proceedings of ACL}, 2024.

\bibitem{cwra2024}
Anonymous.
\newblock Cross-lingual Consistency (CWRA): Translation Attack and X-SIR Defense.
\newblock In {\em Proceedings of ACL}, 2024.

\bibitem{nofreelunch2024}
Anonymous.
\newblock No Free Lunch in LLM Watermarking: Robustness-Usability-Deployability Trilemma.
\newblock In {\em Proceedings of NeurIPS}, 2024.

\bibitem{exploiting2024}
Anonymous.
\newblock Attacking by Exploiting Strengths: Using Public Detection and Quality Preservation as Attack Surface.
\newblock In {\em ICLR Workshop}, 2024.

\bibitem{upv2024}
Anonymous.
\newblock UPV: Unforgeable Publicly Verifiable Watermarking.
\newblock In {\em Proceedings of ICLR}, 2024.

\bibitem{multibit2025}
Anonymous.
\newblock Provably Robust Multi-bit Watermark: Segment-level Pseudo-random Allocation.
\newblock In {\em Proceedings of USENIX Security}, 2025.

\bibitem{stealthink2025}
Anonymous.
\newblock StealthInk: Multi-bit \& Stealth Watermarking without Distribution Change.
\newblock In {\em Proceedings of ICML}, 2025.

\bibitem{multiuser2024}
Anonymous.
\newblock Multi-User Watermarks: Individual/Collusion Group Tracing.
\newblock In {\em IACR ePrint}, 2024.

\bibitem{remark2024}
Ruisheng Zhang, et al.
\newblock REMARK-LLM: Learning-based Encoding-Reparameterization-Decoding Pipeline.
\newblock In {\em Proceedings of USENIX Security}, 2024.
\newblock \url{https://www.usenix.org/conference/usenixsecurity24/presentation/zhang-ruisi}

\bibitem{waterjudge2024}
Anonymous.
\newblock WaterJudge: Quality-Detection Trade-off Evaluation Framework.
\newblock In {\em Findings of NAACL}, 2024.

\bibitem{survey2024}
Anonymous.
\newblock A Survey of Text Watermarking in the Era of Large Language Models.
\newblock {\em ACM Computing Surveys}, 2024.
\newblock \url{https://dlnext.acm.org/doi/pdf/10.1145/3691626}

\bibitem{tutorial2024}
Lei Li, et al.
\newblock Tutorial on LLM Watermarking.
\newblock In {\em Proceedings of ACL Tutorials}, 2024.
\newblock \url{https://aclanthology.org/2024.acl-tutorials.6/}

\end{thebibliography}

\end{document}

